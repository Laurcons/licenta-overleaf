\chapter{Conclusions}

This project was an attempt to prove the usefulness and versatility of Web applications that adhere to the Progressive Web App specification, in contrast with native applications and simple Web applications. To this end, I think it was successful, and it shows that a Web application can have the same feature set as a native application, including having the ability to be installed and ran offline, without going through the hassle of developing separate applications for different mobile operating systems.

Moreover, in spite of existing research on the topic of PWAs, and existing examples of PWA applications, this project stands out in the way that it is not a simple showcase of various Web APIs, but rather a complete application with a feature set that can only reach its full potential in the context of a PWA.

\section{Appropriateness of a PWA architecture}
It is useful to consider, in retrospective, how appropriate the PWA architecture was to this project and its requirements.

If this project had been developed as a normal website, it would have been very easy to access, and a first-time user can get from searching for the app to tracking a train in a very short time. Making the application PWA-compliant would take some time and resources from the app development (in the form of PWA plugin configuration, but also handling of offline scenarios), time which could be used to add more features. However, a website is very ill-suited for the purposes of a trip companion, since it would require a constant Internet connection to make use of trip data.

If this project had been developed as a native application, it would have required separate development processes, one dedicated for iOS and one for Android, in order to have an effective user base that is similar to a website's. It would also require passing through application marketplaces review processes, which, as we have seen, for iOS can be problematic. Although users will be required to download and install the app, which is not a trivial thing to do, users might already expect this and be patient throughout the process. Once installed, the application is now ready to be used in any network conditions. The programmer might find that updating the application can be somewhat hard, since some users might be unwilling to update their applications if it is too inconvenient.

Making the application PWA compliant seems to have resolved the biggest issues from either of the above approaches. Developing for multiple platforms, marketplace approval, or timely updates are not a concern, while the application works under any network conditions.

Therefore, the choice of a PWA architecture is justified.

\section{Potential future work}
The application serves as a useful proof-of-concept of abilities of a PWA application, and it even has potential (in its current form) to be a useful train companion for certain train trips, or for people who travel by train rather rarely.

However, further development opportunities can be identified, some of which are listed below:

\begin{enumerate}
    \item \textbf{Integrating real-time data as received from CFR's official site.} While real-time data from CFR has been considered in this thesis and has been deemed not accurate enough for real-time location tracking, it could prove useful in grounding the application's estimations into the actual reality of the trip.
    \item \textbf{Adding more languages.} While English and Romanian cover a significant part of ridership in Romania, it could be useful to add more languages that tend to other European travelers, such as German or French.
    \item \textbf{Parsing of online tickets.} Online tickets can be purchased for CFR trains and are more convenient than ever before. Although they boast a QR code, the format is not consistent with paper tickets and upon closer inspection does not seem to represent any string data, representing a rather nondescript binary format. However, online tickets come in PDF format, which has potential for different avenues of parsing, in the form of directly looking for text in the PDF file.
    \item \textbf{Parsing of private operator tickets.} Attempting to decode the tickets of private operators could prove as fruitful as decoding CFR physical tickets... or not, but it is an avenue worth exploring.
    \item \textbf{User analytics.} Consenting users could be subjected to event tracking, so as to find out the general usage patterns of features, and make data-backed decisions about the application's interface and functionalities.
\end{enumerate}
\chapter{Evolution of PWAs and capabilities}

\section{Smartphones are everywhere}

The smartphone has seen an incredible rise in ubiquity since its inception. What could once be done only with an expensive desktop computer with a multitude of peripherals, is now available on a comparatively cheap mobile device.

Plotting market shares of desktop, mobile and tablet devices against one another, like shown in figure \ref{FigStatCounterDMT}, it is easy to notice how mobile devices have slowly overtaken desktop devices in popularity in the last 15 years.

\begin{figure}[htbp]
    \centering
    \includegraphics[width=\textwidth]{./figures/ch1_desktop-vs-mobile.png}
    \caption{Market shares of desktop, mobile and tablet devices, from January 2009 to January 2024 \cite{StatCountDMT}. Starting August 1st 2012, the chart counts tablets as a separate metric.}
    \label{FigStatCounterDMT}
\end{figure}

The trend of owning a mobile device shows that people prefer carrying a small and capable device around, rather than only being able to connect to the Internet using their fixed and comparatively large desktop device.

\section{Design differences and similarities between desktop and mobile}

\subsection{Evolution of desktop devices}

It is worth noticing that these two platforms could not be more similar to one another, yet have so many differences. Desktop devices have a notably different evolutionary tree to mobile devices, which means that not only do their shapes and buttons differ, but also their design philosophies.

Computers in their earliest forms were generally thought of as "boiler-room infrastructure" rather than personal devices. The advent of devices like the Apple II and software like VisiCalc, which was an advanced-for-the-time spreadsheet processor released in 1979, started to prove to the world that computers are not only useful when used by a trained team of experts, but can automatize tasks in the reach of a single employee. \cite{NYBirthPC}

These flashy, almost magical machines were starting to be capable of doing various tasks, from boring data processing to running graphics and games. This, combined with how they started to become more and more affordable, resulted in a boom in personal computer ownership starting in the 1980s.

\begin{figure}[htbp]
    \centering
    \includegraphics[width=\textwidth]{./figures/historical_pc_usage.png}
    \caption{Units Shipped per year for various desktop PCs and mobile devices, from 1977 to 2021. \cite{TGPCHis}}
    \label{FigHistoricalPC}
\end{figure}

This direct lineage between the personal computer and pre-1970 mainframes is apparent in the architectures of common operating systems like Windows or Unix. Given how specialized machines, that required trained people to use, started to spread to less technically advanced end-users, it is understandable that computer and operating system vendors tried to simplify how their systems were used.

In particular, the Personal Computer (PC) developed by IBM has seen the longest lifetime of a single lineage of computers \cite{TGPCHis}, being traceable all the way from 1981 to today. This suggests that the evolution of the IBM PC occurred in incremental steps, and to avoid having software vendors be forced to upgrade their software upon release of every version, it is understandable that IBM offered a degree of backwards compatibility.

As a result, design decisions that made sense in the 1980s started to be less useful.


\section{Native applications}

\section{Hybrid (cross-platform interpreted) applications}

\section{Web applications}

\section{Progressive Web Apps}

\section{Comparison}
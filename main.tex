%\documentclass[12pt]{scrreprt}
\documentclass[12pt]{report} 

% language may be romanian or english (default is english)
% type may be bachelor or master (default is bachelor)
\usepackage[language=english, type=bachelor]{style}

%\geometry{a4paper,top=2.5cm,left=3cm,right=2.5cm,bottom=2.5cm}
%in style
%controlling the appearance of your headers and footers
\usepackage{fancyhdr}
\pagestyle{fancy}
\lhead{}
\chead{}
\renewcommand{\headrulewidth}{0.2pt}
\renewcommand{\footrulewidth}{0.2pt}

\begin{document}

\specialization{COMPUTER SCIENCE}	
\title{Providing real-time information on mobile devices, using Progressive Web Apps}
\author{Pricop Laurențiu}
\supervisor{Lect. Dr. Dana Lupșa}
				
\maketitle

\cleardoublepage
SHORT DESCRIPTION (for ELL laboratory only, will be removed)
\vspace{0.5cm}
\hrule
\vspace{0.5cm}

Historically, providing an interface for the user on a mobile phone has presented itself as a choice between two options: a web application, or a native mobile application. However, numerous frameworks have risen up that attempt to bridge the gap and provide the benefits of both worlds. A study by Biørn-Hansen et. al. [1] attempts to compare different technologies for creating cross-platform mobile applications and provides arguments for why Progressive Web Apps (PWAs) are a viable solution for such cross-platform development.

The Romanian national passenger railway operator, CFR Călători SA, provides two methods of viewing real-time information about trains: a website [2] and a mobile app [3], which sit at the opposite poles of the mobile application "spectrum". There is much to gain from combining these two applications into a singular PWA architecture.

In this thesis, in light of lacking literature on the topic of PWAs [4], I attempt to showcase a number of features that PWAs are capable of, that could not be possible in a single application based on either a website or a native mobile app. These features are presented in an application that provides real-time location information about where a passenger is, in relation to their train itinerary, and enables obtaining itinerary information from sources other than manual input, such as ticket QR codes or Optical Character Recognition.

[1]: Biørn-Hansen A., Majchrzak T. and Grønli T. (2017). Progressive Web Apps: The Possible Web-native Unifier for Mobile Development

[2]: \url{https://www.cfrcalatori.ro/cum-cumpar-online/#1517325197560-cf0f95de-6186}

[3]: \url{https://www.cfrcalatori.ro/comunicate/instaleaza-aplicatia-mobila-cfr-calatori-mersul-trenurilor-si-bilete-online-si-poti-cumpara-bilete-de-tren-direct-de-pe-telefon/}

[4]: Mole, Patrick V. (2020). Progressive Web Apps: A Novel Way for Cross-Platform Development

\newpage
\thispagestyle{empty}
\mbox{}
\newpage
\pagenumbering{roman} 

\cleardoublepage
ABSTRACT
\vspace{0.5cm}	
\hrule
\vspace{0.5cm}	
%\cleardoublepage

Abstract: un rezumat în limba engleză cu prezentarea, pe scurt, a conținutului pe capitole, punând accent pe contribuțiile proprii și originalitate



\tableofcontents


\newpage
\pagenumbering{arabic}

\chapter{Introduction}

% Introduction: the paper's objectives and short description of chapters, overview of theme, highlight of personal contribution versus original results, and mention (if applicable) of the communication session where it was presented or the journal where it was published.

\section{Context}

With the rise of mobile devices, users are now able to access many different services directly from their phones. Companies and government agencies are constantly digitizing themselves, and offering their services on mobile devices proves to be the easiest method of reaching their users.

However, methods of delivering applications on mobile phones differ greatly, from phone vendor to another, but also within vendors. Each phone manufacturer provides their own custom way of creating applications for their operating system (Android, iOS). A large-scale study by Haipeng Cai et. al. shows \cite{HaipengAndroidIncomp} that even within Android-based phones exist different flavors of operating systems (by Google, Huawei, Samsung) that are pose incompatibilities to apps, in subtle ways.

As time progressed and the Web evolved, a novel way of creating web applications, that was sitting right on the bridge between native apps and website-based applications, started to form. Starting as early as the year 2000, \verb|XMLHttpRequest| was available on Microsoft's Internet Explorer, and AJAX happened five years later. These, among others, proved to be key parts of the new architecture that designer Frances Berriman and Google engineer Alex Russell coined as "progressive web apps", back in 2015 \cite{PWAShortHist}.

These Progressive Web Apps would advertise themselves as such by using a "web app manifest", and they could register a "service worker" with the browser to put control of the browser's request-response lifecycle in the app's hands \cite{WHATWGWorkers}.

The PWA architecture is described in Alex Russell's words using the following attributes: \cite{PWAShortHist}
\begin{itemize}
    \item \textbf{Responsive:} to fit any form factor;
    \item \textbf{Connectivity independent:} progressively-enhanced with Service Workers to let them work offline;
    \item \textbf{App-like-interactions:} adopt a shell + content application model to create app-like navigation and interactions;
    \item \textbf{Fresh:} transparently always up-to-date thanks to the Service Worker update process;
    \item \textbf{Safe:} served via TLS (a Service Worker requirement) to prevent snooping;
    \item \textbf{Discoverable:} are identifiable as "applications" thanks to W3C manifests and Service Worker registration scope allowing search engines to find them;
    \item \textbf{Re-engageable:} can access the re-engagement UIs of the operating system; e.g. Push Notifications;
    \item \textbf{Installable:} to the home screen through browser-provided prompts, allowing users to "keep" apps they find most useful without the hassle of an app store;
    \item \textbf{Linkable:} meaning they're zero-friction, zero-install, and easy to share.
\end{itemize}

\section{Goals of this thesis}

In this thesis, I will explain the process of creating a Progressive Web Application that can take the role of a train trip companion within Romania's national railway network.

The application possesses all qualities of a PWA, and shows that it can function with minimal internet connection, an issue prevalent in the routes Romania national trains take \cite{CFRInternetIC}, making use of some native capabilities of the mobile device, such as GPS positioning.

The application is able to find the user's location and, using data from their ticket, it displays itinerary information, including what the next station is and estimated arrival time. Ticket data can be obtained either by manual entry, or by scanning the QR code present on the ticket using the phone's camera.

Another showcase of PWA capabilities comes in the form of social login options for the users: they are able to log in using their Google or Apple accounts.

Overall, this thesis attempts to be a showcase of modern Progressive Web App capabilities, all while applying these capabilities to a real problem in the real world.

\section{The real-world problem}
The Romanian national railway operator, CFR S.A. is tasked with the upkeep of 20 thousand kilometers of un-electrified and electrified railway spread out over 9 main lines \cite{CFROrdinMagistrale} \cite{WallstreetRoReorganizareSNCFR}, stretching all across Romania.

\begin{figure}[htbp]
    \centering
    \includegraphics[width=0.8\textwidth]{./figures/ch3_romania-feroviara.png}
    \caption{Map of Romania's railway network. Map by Aero Avalon, shared under CC-BY 4.0 \cite{WikipediaRomaniaFeroviara}.}
    \label{FigRomaniaFeroviara}
\end{figure}

The passenger subsidiary, CFR Călători, is the biggest railway operator that makes use of these lines. However, rolling stock is outdated and most train cars lack modern features like driver-controlled doors. A notable absence though is that of live travel information offered inside the train to the passengers. Examples of such information displays are given in figures \ref{FigMAVInterior}.

\begin{figure}[htbp]
    \centering
    \includegraphics[width=0.8\textwidth]{./figures/ch1_mav-interior.png}
    \caption{Interior of a train owned by MÁV-Start, the Hungarian national passenger operator \cite{PestiHirlapMAVInterior}. The LCD display can be seen at the top, providing information about the next stations, times and delays. Stations and times can be pre-programmed into the train on departure, but delays require live integration with a central service.}
    \label{FigMAVInterior}
\end{figure}

Given that retrofitting all rolling stock with displays connected to the Internet is a challenging task, CFR Călători has attempted digitization in various other ways.

In 2016, the company launched IRIS, a web application that provided traffic information about the company's trains. It also offered information about delays \cite{StiriDeClujLansareCFRIris}. In 2018, IRIS got discontinued in favor of a newer platform, written in ASP.NET, which offers information about all railway operators in Romania.

In 2021, the company launched a mobile application with all the capabilities that the website has, making it more convenient than ever before to view traffic information \cite{MobilissimoCFRLansareMobil}.

These solutions, however, are insufficient in answering the question of "Where am I?" that a passenger might have. The most significant issue is that they all require Internet, the lack of which is an issue prevalent across Romanian trains \cite{CFRInternetIC}.

Moreover, CFR does not track the location of their trains using GPS transponders or any similar technology, but rather using personnel that are responsible with coordinating the trains passing through each station. This brings a number of issues:

\begin{itemize}
    \item \textbf{Data is not real-time.} A train passing through a station is only recorded when the station personnel manually records that train as passed, in the company's internal services.
    \item \textbf{Data might be missing.} Some personnel might not take the time to enter delay data for each passing train.
    \item \textbf{Some stations are not tracked.} Some train stations are big enough to be serviced by regional trains, but not big enough to warrant any personnel responsible with train movement.
    \item \textbf{Data might be false.} Station personnel can enter any delay data they want, having the possibility of hiding delays.
\end{itemize}

\section{Related work}

By having a quick look across Google results for "pwa showcase", one can notice that there exist a number of websites whose sole role is to promote the PWA architecture by showing usage examples of PWA capabilities.

The showcase website created by Joseph Genchik \cite{GenchikPWAShowcase} presents a couple of uses of some phone sensors like Bluetooth, NFC, location, orientation, while also showcasing Web APIs like speech recognition, the Navigator object, web share, WebSockets, and third-party integrations like Auth0 login.

Danny Moerkerke created a website entitled "What PWA Can Do Today" \cite{MoerkerkePWAShowcase} which attempts to be a more feature-complete showcase, making use of diverse Web APIs like shape recognition, barcode detection, protocol handling, notifications API, file system access, storage, background sync, background fetch, payment requests, network information, and speech recognition.

The website at \verb|progressivewebapproom.com| \cite{PWARoom} contains a showcase of a variety of websites and service providers which provide their services using a PWA. Among a multitude of games, notable names include official apps from Tinder, Twitter (now X) and Ride App. It is obvious that these apps are installable due to the fact that the browser gives you the option (sometimes through a pop-up) to install them after first visit.

From looking on the Internet, it is apparent that the PWA architecture is still a novelty among web developers, and used by companies who want to gain some sort of competitive advantage in the user space, by offering varied methods of using their services. In addition, a literature review in the domain of PWAs done by Mole Patrick V. in 2020 \cite{MoleLitRev} suggests that not enough literature exists on the topic of PWAs, and more research is needed around the topic.

%\addcontentsline{toc}{chapter}{Introducere}
% \addcontentsline{toc}{chapter}{Introduction}

\chapter{Evolution of PWAs and capabilities}

\section{Smartphones are everywhere}

The smartphone has seen an incredible rise in ubiquity since its inception. What could once be done only with an expensive desktop computer with a multitude of peripherals, is now available on a comparatively cheap mobile device.

Plotting market shares of desktop, mobile and tablet devices against one another, like shown in Figure \ref{FigStatCounterDMT}, it is easy to notice how mobile devices have slowly overtaken desktop devices in popularity in the last 15 years.

\begin{figure}[htbp]
    \centering
    \includegraphics[width=\textwidth]{./figures/ch2_desktop-vs-mobile.png}
    \caption{Market shares of desktop, mobile and tablet devices, from January 2009 to January 2024 \cite{StatCountDMT}. Starting August 1st 2012, the chart counts tablets as a separate metric.}
    \label{FigStatCounterDMT}
\end{figure}

The trend of owning a mobile device shows that people prefer carrying a small and capable device around, rather than only being able to connect to the Internet using their fixed and comparatively large desktop device.

\section{Design differences and similarities between desktop and mobile}

\subsection{Evolution of desktop devices}

It is worth noticing that these two platforms could not be more similar to one another, yet have so many differences. Desktop devices have a notably different evolutionary tree to mobile devices, which means that not only do their shapes and buttons differ, but also their design philosophies.

Computers in their earliest forms were generally thought of as "boiler-room infrastructure" rather than personal devices. The advent of devices like the Apple II and software like VisiCalc, which was an advanced-for-the-time spreadsheet processor released in 1979, started to prove to the world that computers are not only useful when used by a trained team of experts, but can automatize tasks in the reach of a single employee \cite{NYBirthPC}.

These flashy, almost magical machines were starting to be capable of doing various tasks, from boring data processing to running graphics and games. This, combined with how they started to become more and more affordable, resulted in a boom in personal computer ownership starting in the 1980s.

\begin{figure}[htbp]
    \centering
    \includegraphics[width=\textwidth]{./figures/ch2_historical_pc_usage.png}
    \caption{Units Shipped per year for various desktop PCs and mobile devices, from 1977 to 2021 \cite{TGPCHis}.}
    \label{FigHistoricalPC}
\end{figure}

This direct lineage between the personal computer and pre-1970 mainframes is apparent in the architectures of common operating systems like Windows or Unix. Given how specialized machines, that required trained people to use, started to spread to less technically advanced end-users, it is understandable that computer and operating system vendors tried to simplify how their systems were used.

In particular, the Personal Computer (PC) developed by IBM has seen the longest lifetime of a single lineage of computers (Figure \ref{FigHistoricalPC}), being traceable all the way from 1981 to today. This shows that the evolution of the IBM PC occurred in incremental steps, and to avoid having software vendors be forced to upgrade their software upon release of every version, it is understandable that IBM offered a degree of backwards compatibility.

As a result, design decisions that made sense in the 1980s started to be less useful. A notable example of such design decisions can be found in modern Windows operating systems. The Microsoft documentation \cite{MSDNFolderNaming} has a clause that specifies a number of disallowed directory and file names, the likes of "CON", "PRN", "NUL", "AUX", and others, in addition to any combination of these names with any extension. This is very likely a result of the desire to keep backwards compatibility with old MS-DOS systems, that Windows can be directly traced to, which used these names as aliases for device drivers \cite{HusseinDeviceDrivers}.

\subsection{Evolution of mobile devices}
In comparison, the evolution of mobile devices and smartphones is very distinct and contained in a much shorter period of time.

The first device that was described as "the first time someone created a computer into the shape of a phone" is credited to be the IBM Simon, released around 1992. Boasting a touch screen and being portable, it had features such as a calculator and email. It has been described, retroactively, as "[being] the first smartphone. Twenty years ago, it envisioned our app-happy mobile lives, squeezing the features of a cell phone, pager, fax machine, and computer into an 18-ounce black brick"
\cite{SagerIraHistoryOfPhones}.

The year 1992 is very recent, compared to when the first computers started to become mainstream. This allowed phone manufacturers and OS vendors to learn from the mistakes done by personal computers, or optimize certain user experience facets.

An important step in the evolution of mobile devices is widely regarded as being the first iPhone. The hype Apple created about their new and flashy smartphone, culminating in around 11.000 print articles and 69 million hits on Google, seems to have been justified, and the phone was revolutionary, although somewhat flawed \cite{NYTAppleIPhoneRelease}. The iPhone defined a new era of smartphones, and many phone manufacturers have followed in the path that Apple laid out.

One of the core ideals of the iPhone is the ease of use. People that are non-technical find the iPhone accessible, and the way it seamlessly integrates with other Apple (sadly, only Apple) products proves that technology does not necessarily have to be hard to use \cite{SlashGearIPhoneOverAndroid}.

This trend manifests itself across the entire range of available mobile devices versus personal computers. Mobile phones are more widespread and are generally easier to use, having more friendly user interfaces and abstracting away hard technical details as much as possible.

\section{Native applications}

Mobile devices present ways of installing additional programs called "applications" that can extend the functionality of the device. Applications have proven to be the de-facto standard for implementing mobile functionality, and an ever-increasing number of companies offer their services through the apps that they constantly ask consumers to install.

From an architectural standpoint, these applications are strongly tied to the operating system architecture, and are therefore not portable.

Having said that, an attempt at portability and standardization across mobile devices can be found within the Android Open Source Project. An initiative by Google, the project offers a full mobile operating system under an open-source license, thus allowing any phone maker across the world to use it as a starting point for their own operating system.


\section{Hybrid (cross-platform interpreted) applications}

\section{Web applications}

\section{Progressive Web Apps}

\section{Comparison}
\chapter{Product design}

\section{The real-world problem}
The Romanian national railway operator, CFR S.A. is tasked with the upkeep of 20 thousand kilometers of un-electrified and electrified railway spread out over 9 main lines \cite{CFROrdinMagistrale} \cite{WallstreetRoReorganizareSNCFR}, stretching all across Romania.

\begin{figure}[htbp]
    \centering
    \includegraphics[width=0.8\textwidth]{./figures/ch3_romania-feroviara.png}
    \caption{Map of Romania's railway network. Map by Aero Avalon, shared under CC-BY 4.0 \cite{WikipediaRomaniaFeroviara}.}
    \label{FigRomaniaFeroviara}
\end{figure}

The passenger subsidiary, CFR Călători, is the biggest railway operator that makes use of these lines. However, rolling stock is outdated and most train cars lack modern features like driver-controlled doors. A notable absence though is that of live travel information offered inside the train to the passengers. An example of such an information display is given in Figure \ref{FigMAVInterior}.

\begin{figure}[htbp]
    \centering
    \includegraphics[width=0.8\textwidth]{./figures/ch1_mav-interior.png}
    \caption{Interior of a train owned by MÁV-Start, the Hungarian national passenger operator \cite{PestiHirlapMAVInterior}. The LCD display can be seen at the top, providing information about the next stations, times and delays. Stations and times can be pre-programmed into the train on departure, but delays require live integration with a central service.}
    \label{FigMAVInterior}
\end{figure}

Given that retrofitting all rolling stock with displays connected to the Internet is a challenging task, CFR Călători has attempted digitization in various other ways.

In 2016, the company launched IRIS, a web application that provided traffic information about the company's trains. It also offered information about delays \cite{StiriDeClujLansareCFRIris}. In 2018, IRIS got discontinued in favor of a newer platform, written in ASP.NET, which offers information about all railway operators in Romania.

In 2021, the company launched a mobile application with all the capabilities that the website has, making it more convenient than ever before to view traffic information \cite{MobilissimoCFRLansareMobil}.

These solutions, however, are insufficient in answering the question of "Where am I?" that a passenger might have. The most significant issue is that they all require Internet, the lack of which is an issue prevalent across Romanian trains \cite{CFRInternetIC}.

Moreover, CFR does not track the location of their trains using GPS transponders or any similar technology, but rather using personnel that are responsible with coordinating the trains passing through each station. This brings a number of issues:

\begin{itemize}
    \item \textbf{Data is not real-time.} A train passing through a station is only recorded when the station personnel manually records that train as passed, in the company's internal services.
    \item \textbf{Data might be missing.} Some personnel might not take the time to enter delay data for each passing train.
    \item \textbf{Some stations are not tracked.} Some train stations are big enough to be serviced by regional trains, but not big enough to warrant any personnel responsible with train movement.
    \item \textbf{Data might be false.} Station personnel can enter any delay data they want, having the possibility of hiding delays.
\end{itemize}

\section{Application requirements}
\label{sec:Requirements}

Given the set of real-world issues mentioned above, we can create a set of requirements for an application that tries to plug the information gap and provide a seamless travel experience for the passenger.

The basic idea of this thesis is to merge the mobile app experience and the website experience offered by CFR into a single solution, that will then be extended to offer live traffic information using phone capabilities such as the camera and location.

\begin{enumerate}
    \item Train trip information
          \begin{itemize}
              \item The app should allow the user to create a new trip. The user will be able to view live information about this trip.
              \item The app should be able to show details about the user's current position in their trip, using information obtained from user (train ticket), train database (itinerary of train) and phone sensors (GPS location).
              \item The live information should include: next stop, last stop, next stop arrival time (timetabled), train delay, next station of pass-through (if applicable).
              \item The live information may also include the destination arrival time (time\-tabled) + train delay.
              \item Information about the train ticket can be obtained using multiple methods: scanning the QR code of the ticket (state operator only), or manually inputting ticket data (train number, destination station).
              \item Access to non-live data should be provided via links to the Mersul Trenu\-rilor website (train itinerary, station stops).
              \item National train itinerary information will be saved to the user's device on application install. The information can be updated from the server. Attempts to automatically update this data will be done weekly.
          \end{itemize}
    \item User profile and authentication
          \begin{itemize}
              \item The app should be able to be used anonymously, without login. If so, persistent data must be saved locally to the device.
              \item The app must provide social login (login with Google, login with Apple) as the primary means of logging in.
              \item When creating a new account (logging in for the first time), if there is local data on the device, the app must offer the user the possibility of migrating it to the account. If logging into an existing account, local data must be deleted.
              \item The app must be able to show to the user a history of their trips.
              \item Logging in on another device will log out the already-logged-in device.
          \end{itemize}
    \item Connectivity
          \begin{itemize}
              \item The app must work without Internet, except for functionalities that directly require it.
              \item Account login will not work without Internet.
          \end{itemize}
          \item{Internationalization}
          \begin{itemize}
              \item The app must provide multi-language support (Romanian and English).
          \end{itemize}
\end{enumerate}

\section{Guiding principles}

To maximize user adoption and ease of use, the application should be centered about these core principles:

\begin{itemize}
    \item \textbf{Offline:} it will work at maximum capability while offline.
    \item \textbf{Easy to get into:} users should be able to get live tracking as soon as they click a link, and should not have to install apps.
    \item \textbf{Live information:} the screen will update as soon as the app gets new traffic or position information.
    \item \textbf{Information at a glance:} all information needed to answer the questions of "Where am I?", "How long until next stop?", "How long until my stop?" is always displayed on-screen.
    \item \textbf{Minimal configuration:} integrate with existing CFR services and ticketing systems as seamlessly as possible.
    \item \textbf{Avoid "yet another account and password":} allow the user to save their preferences locally, or use social login to sync them. Do not ask the user to remember another password!
\end{itemize}

\section{Use case diagram}

We can also design a use case diagram showing the main functionalities of the application and how we intend them to interoperate and operate with the various actors, in Figure \ref{FigUseCase}.

\begin{figure}[htbp]
    \centering
    \includegraphics[width=0.9\textwidth]{./figures/ch3_use-case.png}
    \caption{Use case diagram for our CFR Companion application.}
    \label{FigUseCase}
\end{figure}

As you can see, even though we employ user accounts, there is little operational difference between users that are logged in, and those who are not. The permission set is the same, what is different is where the user data (trip history) is saved.

\iffalse
    \section{MOMENTARILY DISCARDED: User stories}

    Given the set of core principles outlined in the previous section, a list of user stories can be compiled. Features that are \textbf{not} necessary for a Minimum Viable Product (MVP) are marked as such and are de-prioritized.

    A minimum viable product, in our case, comprises the set of features needed to fullfil our core principles.

    I used the shorthand "AaU" to mean "As a User".

    \begin{tabularx}
        {\linewidth}{
            | >{\hsize=.15\hsize}X
            | >{\hsize=.65\hsize}X
            | >{\hsize=.2\hsize}X |
        }
        \hline
        Category / Story \# & User story                                                                                                           & Scope        \\
        \hline\hline
        \multicolumn{3}{|X|}{1. Live Location}                                                                                                                    \\
        \hline 1.1.         & AaU, I want to see live where I am, what's the next station, and when I get there                                    & MVP          \\
        \hline 1.2.         & AaU, I want to manually input my train number so that the app knows                                                  & MVP          \\
        \hline 1.3.         & AaU, I want to be able to make a picture of my ticket so that the app can figure out my train number                 & Nice to have \\
        \hline 1.4.         & AaU, I want to be able to scan the ticket's QR code so that the app can figure out my train number                   & Nice to have \\
        \hline 1.5.         & AaU, I want to send a screenshot of an online ticket (or the CFR PDF) so that the app can figure out my train number & Nice to have \\
        \hline 1.6.         & AaU, I want to see, live, what delay my train has                                                                    & MVP          \\
        \hline 1.7.         & AaU, I want to see, live, what the next and last stations are                                                        & MVP          \\
        \hline 1.8.         & AaU, I want to see intermediary stations where the train doesn't stop                                                & Nice to have \\
        \hline 1.9.         & AaU, I want to see, on a map, where I am                                                                             & Nice to have \\
        \hline 1.10.        & AaU, I want to see what station I get off at, and see when I get there                                               & Nice to have \\
        \hline
        \hline
        \multicolumn{3}{|X|}{2. Schedules Information}                                                                                                            \\
        \hline 2.1.         & AaU, I want to see, for a station, what trains arrive there, and at what times                                       & MVP          \\
        \hline 2.2.         & AaU, I want to see, for my train, all stations it has                                                                & MVP          \\
        \hline 2.3.         & AaU, I want to see, for my train, all intermediary stations that it does NOT stop at                                 & Nice to have \\
        \hline
        \hline
        \multicolumn{3}{|X|}{3. User profile and Authentication}                                                                                                  \\
        \hline 3.1.         & AaU, I want to be able to use the app anonymously                                                                    & MVP          \\
        \hline 3.2.         & AaU, if I create an account, I want to be able to merge my data into the account                                     & Nice to have \\
        \hline 3.3.         & AaU, I want to be able to register/login using my Apple account                                                      & MVP          \\
        \hline 3.4.         & AaU, I want to be able to register/login using my Google account                                                     & MVP          \\
        \hline 3.5.         & AaU, I want to be able to optionally set up 2FA using an Authenticator app                                           & Nice to have \\
        \hline 3.6.         & AaU, if I have 2FA on, I want to be able to login using a recovery code                                              & Nice to have \\
        \hline 3.7.         & AaU, I want to add a train trip to my history, manually                                                              & MVP          \\
        \hline 3.8.         & AaU, I want to have any ticket I scan added automatically to my history                                              & Nice to have \\
        \hline 3.9.         & AaU, I want to see the history of my train trips                                                                     & MVP          \\
        \hline 3.10.        & AaU, I want to see a digest of my history: most visited station, most traveled-on route, etc                         & MVP          \\
        \hline
        \hline
        \multicolumn{3}{|X|}{4. Internationalisation}                                                                                                             \\
        \hline 4.1.         & AaU, I want to be able to select a language I prefer to see the app in (EN / RO)                                     & MVP          \\
        \hline
        \hline
        \multicolumn{3}{|X|}{5. Appearance}                                                                                                                       \\
        \hline 5.1.         & AaU, I want to be able to select a dark or light theme                                                               & Nice to have \\
        \hline
    \end{tabularx}
\fi
\chapter{Application architecture}

\section{Architecture considerations}
As a summary of Chapter 2, we can provide a short comparison between possible architectures:

\begin{enumerate}
    \item Native application
          \begin{itemize}
              \item Requires distribution on multiple marketplaces (Google Play Store, Apple App Store).
              \item Requires multiple builds (for each supported platform: iOS, Android).
              \item Installation necessary.
              \item Internet not required for base functionality.
          \end{itemize}
    \item Web application
          \begin{itemize}
              \item Distribution is straightforward (accessing a link is sufficient).
              \item No installation necessary.
              \item Cross-platform by default.
              \item Internet required for base functionality.
          \end{itemize}
    \item Progressive Web Application
          \begin{itemize}
              \item Distribution is straightforward (accessing a link is sufficient).
              \item No installation necessary, although possible.
              \item Internet not required for base functionality (even if not installed).
          \end{itemize}
\end{enumerate}

Remarks relative to our product requirements (section \ref{sec:Requirements}):
\begin{enumerate}
    \item Our train companion app intends to be easy to use and access. Asking users to download and install a native application goes against this intention.
    \item Our train companion app must work without Internet. Asking users to access a website every time they want to use our application goes against this requirement.
\end{enumerate}

The choice of a Progressive Web Application architecture is therefore justified, since this resolves a series of drawbacks of both other possible architectures.

\section{Entities}
By parsing the requirements (section \ref{sec:Requirements}), we can identify a series of entities that the user must interact with:

\begin{enumerate}
    \item Train trip
          \begin{itemize}
              \item A train trip represents a trip defined by a train ticket.
              \item The trip cannot have transfer stops.
              \item A trip has the following properties: train number, departure station, destination station, date, owner (user account). Additional trip information can be deduced from the traffic data (stops, timetable, delays).
          \end{itemize}
    \item User account
          \begin{itemize}
              \item A user account is used to represent a single person using the application.
              \item An account stores the user's preferences.
              \item An account contains a list of social ids, one id for each social login provi\-der (Google, Apple).
          \end{itemize}
\end{enumerate}

\section{Backend services}
The functionalities related to account management and traffic data updates require a backend. This backend needs to be nothing more than a frontend to a database that can store user information (eg. train trip history) and traffic data.

\subsection{Sourcing the traffic data}
CFR Călători submits to the Romanian Government the national train timetables yearly. This data is public, distributed under the \textit{open data} international initiative by the government \cite{DataGovRoDespre}, and is available for use under a variant of the Open Government License \cite{DataGovRoLicense}.

This static data is submitted ahead of every timetabled year, and over the course of a year changes can happen that bring the static data out of date.

To mitigate this issue, the Mersul Trenurilor website provided by CFR can be used \cite{StiriDeClujLansareCFRIris}. Although this data is not shared under an open license, the site's terms and conditions only prohibit commercial use of the information provided, and allow non-commercial use with the notice that the company cannot be held liable for incorrect information.

These two sources of traffic data can be combined in the following manner:
\begin{itemize}
    \item The data.gov.ro dataset can be used yearly to fill in data for all trains.
    \item The Mersul Trenurilor live data can be used to manually adjust the yearly dataset in case of unforeseen events, on a case-by-case basis.
\end{itemize}

The base dataset is provided in a non-standard XML format. Interfacing with this data requires writing a custom parser that can read the 12 Megabyte XML file. Thankfully, this work has already been done by train enthusiast Vasile Coțovanu, with the wonderful Github handle of \textit{vasile}. His work consists of a Ruby parser than can read the non-standard XML and convert it into GTFS information, a format that is easily parsable (comprised of CSV-like text files) and has widespread support. This parser is shared under the MIT license \cite{VasileRubyExporter}.

This GTFS format can then further be parsed and managed using a database management system, to enable amending this data over the course of a year.

\input{chapters/chapter6_conclusions}
%\addcontentsline{toc}{chapter}{Concluzii}
%\addcontentsline{toc}{chapter}{Conclusions}

\bibliography{references}

\end{document}